\addtocontents{toc}{\vspace{1em}}

\chapter{Pendahuluan}

\section{Latar Belakang}
Mengemudi dalam keadaan mengantuk atau \textit{drowsy driving} merupakan ancaman yang membahayakan nyawa pengendara dan pengguna jalan di sekitarnya (Wang dkk., 2017). Di Indonesia, hal ini pernah terjadi di jalan tol Jakarta-Cikampek KM 58 pada 8 April 2024 yang menyebabkan 12 orang tewas (Kompas, 2024).

Dampak dari kejadian yang tidak diinginkan ini meningkatkan kesadaran untuk beristirahat. Salah satu cara untuk memenuhi hal tersebut adalah dibuatnya \textit{rest area} yang terletak di samping jalan tol, sehingga pengendara dapat beristirahat dan melanjutkan perjalanan dalam keadaan segar. Berdasarkan studi yang dilakukan oleh Jung dkk. (2017) di Korea Selatan, adanya tempat istirahat tambahan (\textit{supplemental rest area}) efektif dalam mengurangi jumlah kecelakaan yang disebabkan oleh rasa kantuk sekitar 14\%.

Data mengemudi sulit diperoleh di lingkungan lalu lintas dunia nyata karena potensi bahaya bagi partisipan. Hal ini terutama berlaku untuk studi mengemudi dalam keadaan mengantuk yang mengharuskan partisipan untuk mengantuk agar dapat mengumpulkan data yang bermakna. Memilih lingkungan pengujian yang tepat mungkin sulit, karena lingkungan jalan yang terus berubah membuat tidak mungkin untuk mengisolasi variabel lingkungan binaan yang spesifik (Wang dkk., 2017). Sehingga, simulasi dengan model berbasis agen atau agent-based model (ABM) memungkinkan analisis banyak perilaku individu tanpa perlu data yang diperoleh dari situasi nyata.

ABM telah diterapkan pada berbagai permasalahan yang ada, antara lain pertumbuhan ekonomi pada \textit{rest area} yang ditandai dengan jumlah mobil yang mengunjungi tempat tersebut (Suheri dan Viridi, 2019), durasi pergantian lampu lalu lintas untuk mengurangi jumlah penumpukkan kendaraan pada suatu jalan (Dwitasari, 2019), penggambaran perilaku manusia dalam situasi evakuasi ketika terjadi kebakaran gedung (Firdausyi, 2023), dan evakuasi kebakaran hutan (Siam dkk., 2022).

Dalam penelitian ini, Penulis membuat simulasi dengan pemodelan berbasis agen untuk mengetahui pengaruh rasa kantuk dalam berkendara di jalan tol.

\section{Masalah Penelitian}
Berdasarkan latar belakang yang dijelaskan sebelumnya, masalah pada penelitian ini adalah:
\begin{enumerate}
	\item Bagaimana pengaruh adanya rest area terhadap jumlah kecelakaan di jalan tol yang disebabkan karena rasa kantuk pengendara?
	\item Bagaimana pengaruh variasi jumlah \textit{rest area} dan populasi agen yang digunakan?
\end{enumerate}

\section{Tujuan Penelitian}
Penelitian tesis ini bertujuan untuk:
\begin{enumerate}
	\item Mengetahui adanya rest area terhadap terhadap jumlah kecelakaan di jalan tol yang disebabkan karena rasa kantuk pengendara.
	\item Membandingkan hasil simulasi berdasarkan jumlah {rest area} dan populasi agen yang digunakan.
\end{enumerate}

\section{Batasan Masalah}
Batasan masalah pada penelitian ini adalah sebagai berikut.
\begin{itemize}
	\item Parameter yang digunakan merupakan parameter dengan asumsi.
	\item Karakteristik agen dan lingkungan untuk pemodelan terbatas pada yang didefinisikan pada Bab 3, Metodologi Penelitian.
\end{itemize}

\section{Metode Penelitian}
Penelitian dimulai dari studi teoretis mengenai kecelakaan di jalan tol. Kemudian, dilakukan penulisan kode dan perancangan model berbasis agen dengan menggunakan bahasa pemrograman Python, serta menentukan skenario dan parameter yang akan digunakan pada model.

\section{Sistematika Penulisan}
Laporan ini terdiri dari lima bab. Bab 1 berisi …. Bab 2 berisi …. Bab 3 berisi …. Bab 4 berisi …. Bab 5 berisi ….
