\addtocontents{toc}{\vspace{1em}}

\chapter{Hasil Sementara}

\section{Hasil}

\vspace{1\baselineskip}
\begin{figure}[H]
	\small
	\centering
	\includegraphics[width=1\textwidth]{images/4_test2}
	\caption[Skenario sementara]{Skenario sementara. Ditunjukkan struktur lingkungan berupa rumput (hijau), jalan tol (abu-abu), dan \textit{rest area} (biru langit). Selain itu, juga ditunjukkan beberapa agen dalam kondisi Tidak Mengantuk (biru) dan Sedikit Mengantuk (kuning). Agen yang mengalami kecelakaan dengan keluar jalur (hitam) juga dapat dilihat.}
	\label{fig:4_temp_figure0}
\end{figure}
\vspace{-1\baselineskip}

Berdasarkan hasil yang diperoleh dari simulasi, tidak semua agen yang mempunyai \texttt{drowsiness} $>= 60$ mengunjungi \textit{rest area}, karena tidak berada di lajur yang paling dekat dengan \textit{rest area} dan walaupun mereka berada di lajur tersebut, mereka tetep memiliki peluang untuk melewatinya. Namun, \textit{rest area} memiliki dampak yang positif, karena dapat mengurangi terjadinya kecelakaan di jalan tol. Hal ini dapat dilihat pada Gambar \ref{fig:4_temp_figure}

\vspace{1\baselineskip}
\begin{figure}[H]
	\small
	\centering
	\includegraphics[width=0.6\textwidth]{images/4_t1_accident_combine}
	\caption[Hasil sementara]{Hasil sementara. Ditunjukkan banyaknya kecelakaan dalam simulasi dengan skenario tanpa \textit{rest area} (hitam) dan 1 \textit{rest area} (merah).}
	\label{fig:4_temp_figure}
\end{figure}
\vspace{-1\baselineskip}