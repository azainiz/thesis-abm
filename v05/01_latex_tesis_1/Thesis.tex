\documentclass{itb-thesis}
\graphicspath{{images/}}

% Redefine unit to mm
\makeatletter
\renewcommand*{\lay@value}[2]{\strip@pt\dimexpr0.351459\dimexpr\csname#2\endcsname\relax\relax mm}
\makeatother

% Input Data Tesis
\faculty{Fakultas Matematika dan Ilmu Pengetahuan Alam}

\thesistitle{Simulasi Kelelahan Pengendara Mobil dan Kunjungan ke \textit{Rest Area} dengan \textit{Agent-based Model} (ABM) untuk Mencegah Kecelakaan di Jalan Tol}

\entitle{Simulation of Driver Fatigue and Visits to Rest Areas Using Agent-based Model to Prevent Accidents on Toll Roads}

\fullname{Ahmad Zaini Zahrandika}

\idnum{20923302}

\approvaldate{X Bulan 2026}

\degree{Magister Sains Komputasi}

\monthsubmit{Bulan}

\yearsubmit{2026}

\firstsupervisor{Dr. rer. nat. Sparisoma Viridi, S.Si.}
\firstnip{000001}

\secondsupervisor{Prof. Dr. Second Supervisor}
\secondnip{000002}

\thirdsupervisor{Prof. Dr. Third Supervisor}
\thirdnip{000003}

\begin{document}

% ===================================================== %
% COVER
% ===================================================== %
{
	\cover
}

%\layout

% ===================================================== %
% ABSTRAK
% ===================================================== %
%{
%	\titlespacing*{\chapter}{0pt}{-2.5\baselineskip}{0\baselineskip}
%	\linespread{0.96}\selectfont
%	\abstractid
%	
%	Abstrak yang ditulis dalam Bahasa Indonesia.
%}

% ===================================================== %
% ABSTRACT (BAHASA INGGRIS)
% ===================================================== %
%{
%	\titlespacing*{\chapter}{0pt}{-2.5\baselineskip}{0\baselineskip}
%	\linespread{0.96}\selectfont
%	\abstracten
%	
%	\textit{Abstract that is written in English}
%}

% ===================================================== %
% HALAMAN PENGESAHAN
% ===================================================== %
\approvalpage


% ===================================================== %
% PEDOMAN PENGGUNAAN TESIS
% ===================================================== %
{
	%\fontsize{12pt}{13.8pt}\selectfont
	\thesisguide
	
	Tesis Magister yang tidak dipublikasikan terdaftar dan tersedia di Perpustakaan Institut Teknologi Bandung, dan terbuka untuk umum dengan ketentuan bahwa hak cipta pada penulis dengan mengikuti aturan HaKI yang berlaku di Institut Teknologi Bandung. Referensi kepustakaan diperkenankan dicatat, tetapi pengutipan atau peringkasan hanya dapat dilakukan seizin penulis dan harus disertai dengan kaidah ilmiah untuk menyebutkan sumbernya.
	
	Sitasi hasil penelitian Tesis ini dapat ditulis dalam bahasa Indonesia sebagai berikut:
	\vspace{-3.5em}
	
	\begin{hangparas}{1.27cm}{1}
		Zahrandika, A. (2026): Simulasi Kelelahan Pengendara Mobil dan Kunjungan ke \textit{Rest Area} dengan \textit{Agent-based Model} (ABM) untuk Mencegah Kecelakaan di Jalan Tol, Tesis Program Magister, Institut Teknologi Bandung.
	\end{hangparas}
	
	dan dalam bahasa Inggris sebagai berikut:
	
	\begin{hangparas}{1.27cm}{1}
		Zahrandika, A. (2026): \textit{Simulation of Driver Fatigue and Visits to Rest Areas Using Agent-based Models (ABM) to Prevent Accidents on Toll Roads}, Master's Thesis, Institut Teknologi Bandung.
	\end{hangparas}
	
	Memperbanyak atau menerbitkan sebagian atau seluruh tesis haruslah seizin Dekan Sekolah Pascasarjana, Institut Teknologi Bandung.
	
	Catatan: baris kedua yang merupakan kelanjutan dari baris pertama (satu judul buku), dimulai dengan 7 ketukan (satu Tab) atau ronggak (\textit{hanging indentation}: 1,27 cm) dari tepi halaman.
}


% ===================================================== %
% HALAMAN PERUNTUKAN
% ===================================================== %
%\dedicationpage
%
%\begin{center}
%
%\ 
%\vfill
%\begin{itshape}
%Halaman peruntukan (dedication) bukan halaman yang diharuskan. Jika ada, pada halaman tersebut dituliskan untuk siapa tesis tersebut didedikasikan.
%
%
%Kalimat pada halaman ini diposisikan di bagian tengah kertas.
%
%Contoh
%
%Dipersembahkan kepada orang tua, suami, anak, adik kakak, mertua serta keluarga besarku tercinta yang senantiasa mendukung lahir dan batin.
%\end{itshape}
%\vfill

%\end{center}


% ===================================================== %
% KATA PENGANTAR
% ===================================================== %
%\preface

%Halaman kata pengantar dicetak pada halaman baru. Pada halaman ini mahasiswa S2 berkesempatan untuk menyatakan terima kasih secara tertulis kepada pembimbing dan perorangan lainnya yang telah memberi bimbingan, nasihat, saran dan kritik, serta kepada mereka yang telah membantu melakukan penelitian, kepada perorangan atau badan telah memberi bantuan pembiayaan, dan sebagainya.

%Cara menulis kata pengantar beraneka ragam, tetapi semuanya hendaknya menggunakan kalimat yang baku. Ucapan terima kasih agar dibuat tidak berlebihan dan dibatasi hanya yang "\textit{scientifically related}".


% ===================================================== %
% DAFTAR ISI
% ===================================================== %
{
	\titlespacing*{\chapter}{0pt}{-2.5\baselineskip}{2\baselineskip}
	\linespread{1}\selectfont
	\tableofcontents
}
%\tableofcontents % Daftar Isi (harus ada) [otomatis oleh LaTeX]
\addcontentsline{toc}{chapter}{\contentsname}


% ===================================================== %
% DAFTAR GAMBAR DAN ILUSTRASI
% ===================================================== %
{
	\titlespacing*{\chapter}{0pt}{-2.5\baselineskip}{\baselineskip}
	\linespread{1}\selectfont
	\listoffigures
}
%\listoffigures % Daftar Gambar (umumnya ada) [otomatis oleh LaTeX]
\addcontentsline{toc}{chapter}{\listfigurename}


% ===================================================== %
% DAFTAR TABEL
% ===================================================== %
{
	\titlespacing*{\chapter}{0pt}{-2.5\baselineskip}{\baselineskip}
	\linespread{1}\selectfont
	\listoftables
}
%\listoftables % Daftar Tabel (umumnya ada) [otomatis oleh LaTeX]
\addcontentsline{toc}{chapter}{\listtablename}


% ===================================================== %
% DAFTAR SINGKATAN DAN LAMBANG
% ===================================================== %
\singkatan % Daftar Singkatan (jarang ada)

\begin{singlespace}
\begin{longtable}{@{}p{3cm}@{}@{}p{8cm}@{}@{}p{3cm}@{}}

SINGKATAN & \centering Nama & \raggedright Pemakaian pertama kali pada halaman \\
\endfirsthead
LAMBANG & \centering Nama & \raggedright Pemakaian pertama kali pada halaman \\
\endhead

ABM & \textit{Agent-based Model} &  \tabularnewline
UU & \textit{Undang-Undang} &  \tabularnewline

\end{longtable}
\end{singlespace}

% ===================================================== %

% Pergantian jenis angka untuk nomor halaman
\clearpage
\pagenumbering{arabic}

% ===================================================== %
% BAB 1 - 5
% ===================================================== %

\fontsize{12pt}{13.8pt}\selectfont
%\setlength{\baselineskip}{13.8pt}
\addtocontents{toc}{\vspace{1em}}

\chapter{Pendahuluan}

\section{Latar Belakang}
Mengemudi dalam keadaan mengantuk atau \textit{drowsy driving} merupakan ancaman yang membahayakan nyawa pengendara dan pengguna jalan di sekitarnya (Wang dkk., 2017). Di Indonesia, hal ini pernah terjadi di jalan tol Jakarta-Cikampek KM 58 pada 8 April 2024 yang menyebabkan 12 orang tewas (Kompas, 2024).

Dampak dari kejadian yang tidak diinginkan ini meningkatkan kesadaran untuk beristirahat. Salah satu cara untuk memenuhi hal tersebut adalah dibuatnya \textit{rest area} yang terletak di samping jalan tol, sehingga pengendara dapat beristirahat dan melanjutkan perjalanan dalam keadaan segar. Berdasarkan studi yang dilakukan oleh Jung dkk. (2017) di Korea Selatan, adanya tempat istirahat tambahan (\textit{supplemental rest area}) efektif dalam mengurangi jumlah kecelakaan yang disebabkan oleh rasa kantuk sekitar 14\%.

Data mengemudi sulit diperoleh di lingkungan lalu lintas dunia nyata karena potensi bahaya bagi partisipan. Hal ini terutama berlaku untuk studi mengemudi dalam keadaan mengantuk yang mengharuskan partisipan untuk mengantuk agar dapat mengumpulkan data yang bermakna. Memilih lingkungan pengujian yang tepat mungkin sulit, karena lingkungan jalan yang terus berubah membuat tidak mungkin untuk mengisolasi variabel lingkungan binaan yang spesifik (Wang dkk., 2017). Sehingga, simulasi dengan model berbasis agen atau agent-based model (ABM) memungkinkan analisis banyak perilaku individu tanpa perlu data yang diperoleh dari situasi nyata.

ABM telah diterapkan pada berbagai permasalahan yang ada, antara lain pertumbuhan ekonomi pada \textit{rest area} yang ditandai dengan jumlah mobil yang mengunjungi tempat tersebut (Suheri dan Viridi, 2019), durasi pergantian lampu lalu lintas untuk mengurangi jumlah penumpukkan kendaraan pada suatu jalan (Dwitasari, 2019), penggambaran perilaku manusia dalam situasi evakuasi ketika terjadi kebakaran gedung (Firdausyi, 2023), dan evakuasi kebakaran hutan (Siam dkk., 2022).

Dalam penelitian ini, Penulis membuat simulasi dengan pemodelan berbasis agen untuk mengetahui pengaruh rasa kantuk dalam berkendara di jalan tol.

\section{Masalah Penelitian}
Berdasarkan latar belakang yang dijelaskan sebelumnya, masalah pada penelitian ini adalah:
\begin{enumerate}
	\item Bagaimana pengaruh adanya rest area terhadap jumlah kecelakaan di jalan tol yang disebabkan karena rasa kantuk pengendara?
	\item Bagaimana pengaruh variasi jumlah \textit{rest area} dan populasi agen yang digunakan?
\end{enumerate}

\section{Tujuan Penelitian}
Penelitian tesis ini bertujuan untuk:
\begin{enumerate}
	\item Mengetahui adanya rest area terhadap terhadap jumlah kecelakaan di jalan tol yang disebabkan karena rasa kantuk pengendara.
	\item Membandingkan hasil simulasi berdasarkan jumlah {rest area} dan populasi agen yang digunakan.
\end{enumerate}

\section{Batasan Masalah}
Batasan masalah pada penelitian ini adalah sebagai berikut.
\begin{itemize}
	\item Parameter yang digunakan merupakan parameter dengan asumsi.
	\item Karakteristik agen dan lingkungan untuk pemodelan terbatas pada yang didefinisikan pada Bab 3, Metodologi Penelitian.
\end{itemize}

\section{Metode Penelitian}
Penelitian dimulai dari studi teoretis mengenai kecelakaan di jalan tol. Kemudian, dilakukan penulisan kode dan perancangan model berbasis agen dengan menggunakan bahasa pemrograman Python, serta menentukan skenario dan parameter yang akan digunakan pada model.

\section{Sistematika Penulisan}
Laporan ini terdiri dari lima bab. Bab 1 berisi …. Bab 2 berisi …. Bab 3 berisi …. Bab 4 berisi …. Bab 5 berisi ….

\addtocontents{toc}{\vspace{1em}}

\chapter{Tinjauan Pustaka}

\section{Sistem Kompleks dan \textit{Emergence}}
Sistem kompleks adalah sistem yang perilakunya secara intrinsik sulit untuk dimodelkan karena ketergantungan, persaingan, hubungan, atau jenis interaksi lain di antara elemen-elemennya atau antara sistem tertentu dan lingkungannya (Ladyman dkk., 2013).

Melalui interaksi dari berbagai elemen, sebuah fenomena yang disebut \textit{emergence} muncul yang menjadi karakteristik dari sistem yang kompleks. \textit{Emergence} didefinisikan sebagai kemunculan struktur, pola, dan sifat yang baru dan koheren melalui interaksi beberapa elemen yang tersebar. Struktur yang muncul tidak dapat disimpulkan semata-mata dari sifat-sifat elemen, tetapi juga muncul dari interaksi elemen-elemen tersebut. Struktur yang muncul seperti itu merupakan sifat sistem, namun sering kali memberikan umpan balik kepada elemen-elemen yang menyusunnya (Wilensky dan Rand, 2015).

Ciri-ciri penting dari fenomena \textit{emergence} adalah adanya pola global yang muncul secara spontan dari interaksi berbagai elemen, dan tidak adanya orkestrator atau koordinator yang terpusat. Struktur atau aturan di tingkat mikro atau individu mengarah pada pola yang teratur di tingkat makro. Adanya struktur makro yang terdiri dari banyak elemen, jika terdapat gangguan terhadapnya sering kali menyebabkan struktur tersebut berubah secara dinamis. (Wilensky dan Rand, 2015).

\section{Pemodelan Berbasis Agen}
Pemodelan berbasis agen atau biasa dikenal sebagai \textit{agent-based model} (ABM) adalah model yang biasanya digunakan untuk menganalisis sistem yang kompleks (Janssen dkk., 2019), di mana elemen berupa individu atau agen digambarkan sebagai entitas yang unik dan otonom atau bebas yang biasanya berinteraksi satu sama lain dan lingkungannya secara lokal (Railsback dan Grimm, 2019). ABM juga dapat menghasilkan suatu fenomena \textit{emergence} dari interaksi agen (Macal dan North, 2010).

Beberapa karakteristik dari ABM adalah sebagai berikut (Wilensky dan Rand, 2015).
\begin{enumerate}
	\item Pada umumnya, pemodelan berbasis persamaaan atau equation-based model (EBM) menggunakan asumsi homogenitas, sedangkan ABM dapat dilakukan untuk populasi yang heterogen atau beragam.
	\item Interaksi dan hasil dari simulasi ABM bersifat diskrit. Sebagai contoh, dinamika populasi yang menggunakan EBM memperlakukan populasi seolah-olah mereka bersifat kontinu padahal sebenarnya populasi adalah kumpulan individu-individu yang terpisah-pisah.
	\item Dalam penerapan ABM, tidak diperlukan pemahaman yang mendalam mengenai pola keseluruhan yang akan dihasilkan dari perilaku masing-masing individu. Sebaliknya, diperlukan pemahaman perilaku individu untuk menentukan aturan-aturan pada entitas yang digunakan, sehingga dapat memberikan hasil yang dapat diamati melalui simulasi. Bahkan jika tidak memiliki hipotesis tentang bagaimana variabel agregat akan berinteraksi, model masih dapat dibangun dan hasil dapat diperoleh.
	\item Karena ABM menggambarkan individu, bukan keseluruhan, hubungan antara model dan dunia nyata lebih cocok. Oleh karena itu, akan lebih mudah untuk menjelaskan apa yang dilakukan oleh sebuah model kepada seseorang yang tidak memiliki pelatihan dalam paradigma pemodelan tertentu. 
	\item ABM dapat memberikan detail tingkat individu dan agregat pada saat yang bersamaan. Karena ABM memodelkan setiap individu dan keputusan mereka, maka dimungkinkan untuk memeriksa riwayat dan kehidupan setiap individu dalam model, atau seluruh individu dan mengamati hasil keseluruhan. Pendekatan ini disebut “\textit{bottom-up}”.
	\item Mudah untuk memasukkan sifat acak atau \textit{randomness} ke dalam model karena keputusan yang akan diambil oleh agen dapat dibuat berdasarkan probabilitas.
\end{enumerate}

\vspace{3em}

Perlu diketahui bahwa setiap metode memiliki keterbatasan, dan ABM tidak terkecuali. Salah satunya adalah membutuhkan daya komputasi yang besar karena memodelkan banyak individu atau agen dalam waktu yang sama (Wilensky dan Rand, 2015).

Terdapat beberapa komponen yang berperan penting dalam ABM, yaitu agen, lingkungan, dan interaksi (Macal dan North, 2010; Wilensky dan Rand, 2015).

Agen merupakan entitas dasar dalam ABM, sehingga penting untuk merancang agen dengan baik. Dua aspek yang penting dalam mendefinisikan agen adalah properti atau ciri-ciri yang dimiliki serta perilaku atau tindakan yang dapat dilakukan. Properti agen menggambarkan keadaan suatu agen, sedangkan tindakan atau perilaku agen adalah cara-cara di mana suatu agen dapat mengubah keadaan lingkungan, agen lain, atau dirinya sendiri (Wilensky dan Rand, 2015). Selain itu, parameter agen dapat bersifat diskrit (seperti jenis kelamin) atau kontinu (seperti pendapatan) (McDonald dan Osgood, 2023).

Lingkungan terdiri dari kondisi dan habitat yang mengelilingi agen saat mereka bertindak dan berinteraksi dalam model. Lingkungan dapat memengaruhi keputusan agen dan sebaliknya (Wilensky dan Rand, 2015).

Untuk interaksi, terdapat lima jenis interaksi yang ada pada ABM (Wilensky dan Rand, 2015), yaitu
\begin{enumerate}
	\item \textit{Agent-Self Interactions}, yaitu agen interaksi dengan dirinya sendiri. Agen mempertimbangkan keadannya saat ini dan menentukan apa yang dilakukannya. Contoh dari interaksi ini adalah birth atau kelahiran agen baru dan death atau kematian agen yang telah ada.
	\item \textit{Environment-Self Interactions}, yaitu ketika lingkungan interaksi dengan dirinya sendiri. Contoh dari interaksi ini adalah suatu rumput yang tumbuh.
	\item \textit{Agent-Agent Interactions}, yaitu interaksi antara dua agen atau lebih. Contoh dari interaksi ini adalah predator dan mangsa, di mana agen predator memburu agen mangsa serta agen predator bersaing dengan predator lainnya untuk mendapatkan mangsa.
	\item \textit{Environment-Environment Interactions}, yaitu interaksi antara berbagai bagian lingkungan yang berbeda. Contoh dari interaksi ini adalah difusi.
	\item \textit{Agent-Environment Interactions}, yaitu interaksi yang terjadi ketika agen memanipulasi atau menguji bagian lingkungan tempatnya berada, atau ketika lingkungan berubah ketika mengobservasi agen yang berada di dalamnya. Contoh dari interaksi ini adalah semut yang mencari makanan, di mana semut perlu mengobservasi lingkungannya untuk mencari lingkungan atau tempat yang berpotensi menjadi sumber makanan, sehingga makanan dapat diambil kemudian.
\end{enumerate}

\section{Contoh Pemanfaatan ABM}

\subsection{Simulasi Pertumbuhan Ekonomi di Sepanjang Jalan Tol}

Penelitian yang bertujuan untuk mengetahui pertumbuhan ekonomi di sepanjang jalan tol yang ditandai dengan jumlah mobil yang mengunjungi \textit{rest area} pernah dilakukan oleh Tatang dan Viridi (2019) dengan ABM yang digabungkan dengan model gravitasi atau \textit{gravity model} (GM).

\vspace{1em}
\begin{figure}[H]
	\small
	\centering
	\includegraphics[width=1\textwidth]{images/2_toll_ex1}
	\caption[Gambar sederhana 1]{Sistem simulasi oleh Tatang dan Viridi (2019) yang terdiri dari beberapa bagian: (A) kota A, (B) kota B, (C) jalan tol dari A ke B, (D) jalan tol dari B ke A, (E) rest area di jalan dari A ke B, dan (F) rest area di jalan dari B ke A, dan (G) dua jenis agen yang berbeda (Tatang dan Viridi, 2019).}
	\label{fig:2_toll_ex1}
\end{figure}
\vspace{-2em}

Karakteristik simulasi tersebut adalah sebagai berikut:
\begin{itemize}
	\item Beberapa aturan-aturan yang berlaku untuk para agen pada ABM diturunkan dari GM. 
	\item Agen terdiri dari dua jenis, yaitu yang tidak butuh mengunjungi \textit{rest area} (biru), dan yang butuh (merah).
	\item Pergerakan agen secara diskrit atau menggunakan grid.
	\item Variasi skenario berdasarkan banyaknya agen berwarna merah dan biru.
%	\item Simulasi dibuat dengan bahasa pemrograman JavaScript.
\end{itemize}

\vspace{2.5\baselineskip}
\begin{figure}[H]
	\small
	\centering
	\includegraphics[width=1\textwidth]{images/2_toll_ex2}
	\caption[Gambar sederhana 2]{Pergerakan agen-agen pada simulasi. Hanya ada beberapa agen berwarna merah yang berada di dalam rest area (Tatang dan Viridi, 2019).}
	\label{fig:2_toll_ex2}
\end{figure}
\vspace{-1\baselineskip}

Hasil dari penelitian yang dilakukan oleh Tatang dan Viridi (2019) dengan variasi perbandingan antara agen yang berwarna biru dan yang berwarna merah, adalah semakin banyak agen merah, maka semakin sedikit jumlah agen yang mengunjungi kota A dari kota B dan sebaliknya, tetapi jumlah agen yang mengunjungi \textit{rest area} juga bertambah. Selain itu, \textit{rest area} yang dekat dengan kota asal suatu agen juga memiliki lebih banyak pengunjung daripada \textit{rest area} yang lokasinya jauh dengan kota asal.

\subsection{Simulasi Evakuasi Gedung}

\vspace{0.5\baselineskip}
\begin{figure}[H]
	\small
	\centering
	\includegraphics[width=0.55\textwidth]{images/2_evac_ex1}
	\caption[Gambar sederhana 3]{Ilustrasi lingkungan berupa interior suatu gedung pada simulasi oleh Firdausyi (2023) yang terdiri dari dinding (coklat), titik keluar (hijau tua), serta agen yang mewakili anak-anak (hijau muda), orang dewasa (kuning), manusia lanjut usia (jingga), dan manusia dengan disabilitas (biru tua) (Firdausyi, 2023).}
	\label{fig:2_evac_ex1}
\end{figure}
\vspace{-1\baselineskip}

Pada tahun 2023, Firdausyi mempublikasikan penelitian yang berkaitan dengan simulasi evakuasi ketika terjadi kebakaran gedung. Metode yang digunakan dalam pembuatan simulasi tersebut adalah ABM digunakan untuk merepresentasikan karakteristik dan pengambilan keputusan masing-masing individu, sedangkan sedangkan model gaya sosial atau \textit{social force model} (SFM) digunakan untuk menggambarkan pergerakan agen.

Karakteristik simulasi tersebut adalah sebagai berikut.
\begin{itemize}
	\item Agen terdiri dari empat jenis, yaitu yang mewakili anak-anak (warna hijau muda), orang dewasa (warna kuning), manusia lanjut usia (warna jingga), dan manusia dengan disabilitas (warna biru tua).
	\item Pergerakan agen secara kontinu.
	\item Variasi skenario berdasarkan kepadatan populasi agen, jumlah titik keluar, dan lebar titik keluar.
%	\item Simulasi dibuat dengan NetLogo.
\end{itemize}

\vspace{1\baselineskip}
\begin{figure}[H]
	\small
	\centering
	\includegraphics[width=0.5\textwidth]{images/2_evac_ex2}
	\caption[Gambar sederhana 4]{Pergerakan agen-agen yang menuju titik keluar gedung (Firdausyi, 2023).}
	\label{fig:2_evac_ex2}
\end{figure}
\vspace{-1\baselineskip}

Hasil yang diperoleh pada penelitian yang dilakukan oleh Firdausyi (2023) adalah muncul perilaku kolektif atau emergence seperti sumbatan (\textit{bottleneck}), \textit{arching}, dan \textit{clogging} ketika agen bergerak menuju titik keluar dengan segera. Selain itu, semakin banyak agen yang berada pada sekitar titik keluar, maka akan semakin cepat proses evakuasi berlangsung karena jarak yang harus ditempuh menuju titik keluar semakin pendek. Kemudian, peletakan titik keluar yang lebih efektif mempersingkat durasi evakuasi adalah ketika titik keluar diletakkan di sisi yang berbeda satu sama lain. Terakhir, lebar titik keluar dapat memengaruhi durasi evakuasi, hingga pada lebar tertentu perbedaannya tidak lagi signifikan.

\subsection{Simulasi Evakuasi Kebakaran Hutan}

\vspace{0.5\baselineskip}
\begin{figure}[H]
	\small
	\centering
	\includegraphics[width=1\textwidth]{images/2_evac2_ex1}
	\caption[Gambar sederhana 5]{Model evakuasi kebakaran hutan berbasis agen di Mati, Yunani di NetLogo. Panel kiri berisi daftar variabel sistem yang dapat dikontrol seperti pembagian mode dan strategi evakuasi, serta aspek perilaku dan fisik dari penduduk di area beresiko. Dinamika evakuasi ditampilkan dalam tampilan di panel tengah, yang menunjukkan pergerakan mobil dan pejalan kaki melalui \textit{evacuation route system}, bersama dengan perambatan api (warna jingga) dan area prioritas yang dipilih untuk evakuasi bertahap (bentuk geometris berwarna merah muda). Panel kanan berisi hasil evakuasi dalam hal jumlah korban kebakaran (Siam dkk., 2022).}
	\label{fig:2_evac2_ex1}
\end{figure}
\vspace{-1\baselineskip}

ABM juga diterapkan untuk pada simulasi evakuasi kebakaran hutan yang terjadi pada tahun 2018 di desa Mati, Yunani. (Siam dkk., 2022). Model ini mengintegrasikan sistem bahaya alam (penyebaran api) dengan sistem respons sosioteknis yang terdiri dari komponen sosial (respons penduduk) dan komponen rekayasa (jaringan transportasi dan lokasi penampungan). Tujuan penelitian tersebut adalah untuk menginvestigasi dampak dari keputusan penduduk di daerah beresiko terhadap korban kebakaran hutan dan lahan mengenai apakah mereka akan pergi dan berapa lama mereka harus menunggu (misalnya, waktu keberangkatan), pilihan transportasi yang digunakan, seperti jalan kaki atau berkendara, dan seberapa cepat mereka melakukan perjalanan.

\vspace{1\baselineskip}
\begin{figure}[H]
	\small
	\centering
	\includegraphics[width=0.8\textwidth]{images/2_evac2_ex2}
	\caption[Gambar sederhana 6]{Cuplikan simulasi model pada berbagai tahap kejadian kebakaran hutan, pada (a) $t=0 \, min$ menunjukkan di mana populasi awal dan tempat penampungan terdistribusi; (b) $t=40 \, min$ menunjukkan pergerakan pengungsi baik dengan mobil (biru muda) atau berjalan kaki (ungu); dan (c) - (d) kebakaran hutan melanda kota sekitar $t=80 \, min$ hingga $t=180 \, min$, yang menyebabkan korban jiwa (jingga) (Siam dkk., 2022).}
	\label{fig:2_evac2_ex2}
\end{figure}
\vspace{-1\baselineskip}

Hasil yang diperoleh dari penelitian yang dilakukan oleh Siam dkk. (2022) adalah sebagai berikut:
\begin{itemize}
	\item Baik kecepatan berjalan kaki pejalan kaki maupun kecepatan kendaraan mengurangi tingkat kematian, tetapi efek dari variabel-variabel ini bergantung pada nilai variabel lain.
	\item Pesan yang ditargetkan dalam bentuk diskusi kelompok terarah atau strategi lain yang sesuai harus diberikan kepada rumah tangga yang memiliki anak-anak untuk mempromosikan perencanaan evakuasi sebelum kejadian.
	\item Evakuasi bertahap atau staged evacuation dapat mengurangi angka kematian secara signifikan dibandingkan dengan evakuasi serentak.
	\item Kematian terendah dapat dicapai ketika ada pemisahan moda (modal split) antara evakuasi kendaraan dan pejalan kaki. Dalam kasus ini, hal ini terjadi ketika 70\% pengungsi memilih berjalan kaki dan 30\% memilih berkendara.
	\item Skema alokasi kapasitas tempat penampungan yang lebih baik dapat secara signifikan mengurangi angka kematian.
\end{itemize}

\section{Jalan Tol}
Menurut Peraturan Pemerintah Nomor 15 Tahun 2005, jalan tol adalah adalah jalan umum yang merupakan bagian sistem jaringan jalan dan sebagai jalan nasional yang penggunanya diwajibkan membayar tol. Penyelengaraan jalan tol bertujuan meningkatkan efisiensi pelayanan jasa distribusi guna menunjang peningkatan pertumbuhan ekonomi terutama di wilayah yang sudah tinggi tingkat perkembangannya.

\section{\textit{Rest Area}}
Berdasarkan Peraturan Menteri Pekerjaan Umum dan Perumahan Rakyat Republik Indonesia Nomor 28 Tahun 2021, \textit{rest area} adalah suatu tempat istirahat yang yang dilengkapi dengan berbagai fasilitas umum bagi pengguna jalan tol, sehingga baik bagi pengendara, penumpang, maupun kendaraannya dapat beristirahat untuk sementara. Rest area juga dilengkapi dengan berbagai fasilitas, antara lain tempat parkir, miniswalayan, peturasan atau tempat buang air kecil, dan stasiun pengisian bahan bakar umum.
\addtocontents{toc}{\vspace{1em}}

\chapter{Metodologi Penelitian}

\section{Pembuatan Model}
Terdapat beberapa langkah dalam penelitian ini. Pertama, pembuatan program simulasi dengan bahasa pemrograman Python. Pendekatan simulasi yang digunakan adalah model berbasis agen atau \textit{agent-based model} (ABM). Verifikasi juga akan dilakukan untuk menghilangkan adanya kesalahan pada kode program yang dibuat. Dalam tahap ini, beberapa uji coba akan dilakukan untuk memastikan bahwa kode dari simulasi sesuai dengan fenomena yang diinginkan. Cara verifikasi yang akan dilakukan adalah dengan mengubah nilai input suatu model, memvariasikan parameter-parameter yang dimiliki pada model, menambahkan parameter dalam suatu model, atau mengubah lingkungan yang digunakan dalam model. Simulasi akan dilakukan secara berulang dengan berbagai skenario yang memiliki parameter yang berbeda, seperti banyaknya jumlah agen dan \textit{rest area}.

\subsection{Penggunaan Perangkat Lunak dalam Pelaksanaan Penelitian}
Dalam penelitian ini, perangkat lunak yang digunakan meliputi TeXstudio sebagai pengolah kata untuk dokumentasi penelitian tesis ini, dan Python sebagai bahasa pemrograman utama. Python dipilih karena ketersediaan \textit{library} atau modul yang mendukung pembuatan model berbasis agen. Modul yang dilibatkan adalah NumPy dan Math untuk operasi numerik, serta Matplotlib untuk visualisasi data.

\subsection{Lingkungan dan Waktu}
Dalam penelitian ini, digunakan \textit{grid} dua dimensi sebagai kerangka dasarnya. \textit{Grid} yang digunakan direpresentasikan dalam koordinat kartesian dengan sumbu-$x$ dan $y$. Setiap agen dan \textit{rest area} menempati satu sel pada \textit{grid}, sedangkan komponen lingkungan berupa jalan tol dapat menempati lebih dari satu sel pada \textit{grid}.

\vspace{1\baselineskip}
\begin{table}[H]
	\centering
	\begin{tabular}{|l|l|l|l|l|}
		\hline
		\textbf{Struktur}  & \textbf{Deskripsi}                                                          & \textbf{Warna} & \textbf{Nilai} & \textbf{Karakteristik}                                                                \\ \hline
		Rumput             & \begin{tabular}[c]{@{}l@{}}Untuk membatasi \\ ruang gerak agen\end{tabular} & Hijau          & $0$ & \begin{tabular}[c]{@{}l@{}}Tidak dapat \\ ditempati oleh agen\end{tabular} \\ \hline
		Jalan Tol          & \begin{tabular}[c]{@{}l@{}}Tempat agen \\ bergerak\end{tabular}             & Abu-abu & $1$     & \begin{tabular}[c]{@{}l@{}}Dapat ditempati \\ oleh agen\end{tabular} \\ \hline
		\textit{Rest Area} & \begin{tabular}[c]{@{}l@{}}Tempat agen untuk \\ beristirahat\end{tabular}   & Biru langit & $2$ & \begin{tabular}[c]{@{}l@{}}Dapat ditempati \\ oleh agen\end{tabular} \\ \hline
	\end{tabular}
	\centeredcaption{Deskripsi struktur lingkungan yang digunakan.}
	\label{tab:3_env_structure}
\end{table}
\vspace{-1\baselineskip}

Setiap proses simulasi dimulai dengan menentukan panjang dan lebar dalam sumbu-$x$ dan $y$. Kemudian, dilakukan pengaturan jalan tol, seperti panjang dan jumlah lajurnya, dan \textit{rest area} untuk membentuk lingkungan yang akan digunakan. Pada tahap ini, struktur berupa jalan tol dibentuk untuk menggambarkan suatu ruang tempat agen berada. Deskripsi dari struktur serta visualisasi lingkungan yang digunakan dapat dilihat pada Tabel \ref{tab:3_env_structure} dan Gambar \ref{fig:3_env_grid_example}.

\vspace{0.5\baselineskip}
%\vspace{2.5\baselineskip}
\begin{figure}[H]
	\small
	\centering
	\includegraphics[width=1\textwidth]{images/3_env_example}
	\caption[Gambar sederhana 3]{Contoh lingkungan yang digunakan. Warna hijau, abu-abu, dan biru langit masing-masing mempresentasikan  rumput, jalan tol, dan \textit{rest area}.}
	\label{fig:3_env_grid_example}
\end{figure}
\vspace{-1\baselineskip}

\subsection{Pergerakan Agen}
Dalam penelitian ini, pergerakan agen yang digunakan adalah diskrit, karena

\subsection{Properti Agen}
Pada simulasi ini, agen merepresentasikan seorang pengendara mobil yang terlibat dalam suatu perjalanan. Agen memiliki beberapa variabel, posisi dalam koordinat kartesian $x$ dan $y$ \textit{drowsiness}. Informasi lebih lanjut terkait variabel agen dapat dilihat pada Tabel \ref{tab:3_agent_variables}

%\vspace{2.5\baselineskip}
\begin{table}[H]
	\centering
	\begin{tabular}{|l|l|l|}
		\hline
		\textbf{Variabel}  & \textbf{Deskripsi}                                                                                                                                                                            & \textbf{Nilai} \\ \hline
		$x$ dan $y$        & Koordinat posisi agen di dalam lingkungan                                                                                                                                                     & $\mathbb{Z} \geq 0$              \\ \hline
		\textit{drowsiness}    & Tingkat rasa kantuk suatu agen                                                                                                                                                                & $\mathbb{Z} >\geq 0$      \\ \hline
		\textit{direction} & Arah gerak suatu agen                                                                                                                                                                          & $[1, 3]$       \\ \hline
		\textit{status}     & \begin{tabular}[c]{@{}l@{}}Status yang berkaitan dengan suatu agen, \\ 
			yaitu berkendara di jalan tol, berada di \\ \textit{rest area}, sudah sampai tujuan, dan\\ mengalami kecelakaan\end{tabular} & $[1, 4]$       \\ \hline
	\end{tabular}
	\centeredcaption{Deskripsi struktur lingkungan yang digunakan.}
	\label{tab:3_agent_variables}
\end{table}
\vspace{-1\baselineskip}

Tabrakan didefinisikan sebagai dua agen yang menempati sel \textit{grid} di lajur jalan tol yang sama, sehingga tabrakan tidak dianggap sebagai \textit{n-body}.

\subsection{Perilaku Agen}

Selain itu, agen dibagi ke dalam tiga kategori berdasarkan tingkat kantuk atau \textit{drowsiness}, yaitu tidak mengantuk, sedikit mengantuk, dan mengantuk. Setiap kategori memiliki 

Terdapat beberapa interaksi yang terjadi pada suatu agen:
\begin{itemize}
	\item Agen berinteraksi dengan dirinya sendiri. Setiap agen memiliki variabel tingkat kelelahan atau \textit{drowsiness} yang memengaruhi pengambilannya keputusan.
	\item Agen berinteraksi dengan agen lainnya. Agen mengamati agen lain di sekitarnya dalam jarak tertentu, yang memengaruhi keputusan arah gerak.
	\item Agen berinteraksi dengan lingkungan. Agen berinteraksi dengan elemen lingkungan jalan, seperti jumlah lajur dan lokasi \textit{rest area}. Keberadaan \textit{rest area} akan memicu agen dengan tingkat kelelahan tinggi untuk mempertimbangkan beristirahat.
\end{itemize}
\vspace{2.5\baselineskip}

\begin{figure}[H]
	\small
	\centering
	\includegraphics[width=1\textwidth]{images/3_flowchart_movement_decision}
	\caption[Gambar sederhana 3]{\textit{Flowchart} untuk menentukan kemungkinan arah gerak suatu agen.}
	\label{fig:3_flow_movement}
\end{figure}
\vspace{-1\baselineskip}

Selain itu, terdapat aturan
\begin{enumerate}
	\item Jika tidak ada jalan di depan (jalan buntu) atau ada agen lain di depan agen ini, maka pilihan gerak yang layak hanyalah pindah ke lajur kiri atau kanan.
	\item Jika lajur di kiri tidak tersedia, baik ditempati oleh agen lainnya atau tidak ada, pilihan gerak yang layak adalah tetap lurus atau pindah ke lajur kanan.
	\item Jika lajur di kanan tidak tersedia, baik ditempati oleh agen lainnya atau tidak ada, maka pilihan gerak yang layak adalah tetap lurus atau pindah ke lajur kiri.
	\item Jika tingkat kantuk agen adalah Sangat Mengantuk (>100), maka aturan nomor 1-3 diabaikan.
	\item Jika ada \textit{rest area} di sebelah kiri dan status agen adalah Sedikit Mengantuk (60-100) atau Sangat Mengantuk (>100), maka ada probabilitas 75\% untuk belok kiri ke \textit{rest area}.
	\item Jika tidak ada satupun arah (lurus, kiri, kanan) yang layak atau mengalami kecelakaan (keluar dari jalan tol atau tabrakan dengan agen lain), agen harus berhenti dan tetap berada di \textit{grid} sebagai penghalang selama $N$ langkah waktu simulasi. Setelah $N$ langkah waktu berlalu, agen dihapus ("menghilang") dari simulasi.
\end{enumerate}
\addtocontents{toc}{\vspace{1em}}

\chapter{Hasil Sementara}

\section{Hasil}

\vspace{1\baselineskip}
\begin{figure}[H]
	\small
	\centering
	\includegraphics[width=1\textwidth]{images/4_test2}
	\caption[Skenario sementara]{Skenario sementara. Ditunjukkan struktur lingkungan berupa rumput (hijau), jalan tol (abu-abu), dan \textit{rest area} (biru langit). Selain itu, juga ditunjukkan beberapa agen dalam kondisi Tidak Mengantuk (biru) dan Sedikit Mengantuk (kuning). Agen yang mengalami kecelakaan dengan keluar jalur (hitam) juga dapat dilihat.}
	\label{fig:4_temp_figure0}
\end{figure}
\vspace{-1\baselineskip}

Berdasarkan hasil yang diperoleh dari simulasi, tidak semua agen yang mempunyai \texttt{drowsiness} $>= 60$ mengunjungi \textit{rest area}, karena tidak berada di lajur yang paling dekat dengan \textit{rest area} dan walaupun mereka berada di lajur tersebut, mereka tetep memiliki peluang untuk melewatinya. Namun, \textit{rest area} memiliki dampak yang positif, karena dapat mengurangi terjadinya kecelakaan di jalan tol. Hal ini dapat dilihat pada Gambar \ref{fig:4_temp_figure}

\vspace{1\baselineskip}
\begin{figure}[H]
	\small
	\centering
	\includegraphics[width=0.6\textwidth]{images/4_t1_accident_combine}
	\caption[Hasil sementara]{Hasil sementara. Ditunjukkan banyaknya kecelakaan dalam simulasi dengan skenario tanpa \textit{rest area} (hitam) dan 1 \textit{rest area} (merah).}
	\label{fig:4_temp_figure}
\end{figure}
\vspace{-1\baselineskip}
%\addtocontents{toc}{\vspace{1em}}

\chapter{Simpulan}

% ===================================================== %
% Daftar Pustaka, Lampiran
% ===================================================== %

\addtocontents{toc}{\vspace{1em}}
\addcontentsline{toc}{chapter}{DAFTAR PUSTAKA}

\setlength{\parskip}{0em}
\linespread{0.96}\selectfont
\titlespacing*{\chapter}{0pt}{-1\baselineskip}{1\baselineskip}

\chapter*{DAFTAR PUSTAKA}

\begin{hangparas}{1.27cm}{1}
	Dwitasari, M. (2019): \textit{Durasi Pergantian Lampu Lintas untuk Mencapai Jumlah Minimum Penumpukkan Kendaraan dengan Agent-Based Model}, Tesis Program Master, Institut Teknologi Bandung.
\end{hangparas}

\begin{hangparas}{1.27cm}{1}
	Firdausyi, A. M. (2023): \textit{Pemodelan Evakuasi Pedestrian dengan Metode Agent-Based Model dan Social-Force Model}, Tesis Program Master, Institut Teknologi Bandung.
\end{hangparas}

\begin{hangparas}{1.27cm}{1}
	Indonesia (2004): Undang-Undang Republik Indonesia Nomor 38 Tahun 2004 tentang Jalan.
\end{hangparas}

\begin{hangparas}{1.27cm}{1}
	Indonesia (2005): Peraturan Pemerintah Republik Indonesia Nomor 15 Tahun 2005 tentang Jalan Tol.
\end{hangparas}

\begin{hangparas}{1.27cm}{1}
	Janssen, S., Sharpanskykh, A., Curran, R., dan Langendoen, K. (2019): Using causal discovery to analyze emergence in agent-based models, \textit{Simulation Modelling Practice and Theory}, \textbf{96}, 101940. https://doi.org/10.1016/j.simpat.2019.101940.
\end{hangparas}

\begin{hangparas}{1.27cm}{1}
	Jung, S., Joo, S., dan Oh, C. (2017): Evaluating the effects of supplemental rest areas on freeway crashes caused by drowsy driving, \textit{Accident Analysis \& Prevention}, \textbf{99}, 356–363. https://doi.org/10.1016/j.aap.2016.12.021.
\end{hangparas}

\begin{hangparas}{1.27cm}{1}
	Kompas (15 April 2024): Kesimpulan Polri, Kecelakaan Tol Cikampek karena Sopir Gran Max Kelelahan, Berisiko “Microsleep”, diperoleh dari situs internet:
	https://nasional.kompas.com/read/2024/04/15/15300411/kesimpulan-polri-\\kecelakaan-tol-cikampek-karena-sopir-gran-max-kelelahan.
\end{hangparas}

\begin{hangparas}{1.27cm}{1}
	Ladyman, J., Lambert, J., dan Wiesner, K. (2013): What is a complex system?, \textit{European Journal for Philosophy of Science}, \textbf{3}(1), 33–67. https://doi.org/10.1007/s13194-012-0056-8.
\end{hangparas}

\begin{hangparas}{1.27cm}{1}
	Macal, C. M., dan North, M. J. (2010): Tutorial on agent-based modelling and simulation, Journal of Simulation, 4(3), 151–162. https://doi.org/10.1057/jos.2010.3.
\end{hangparas}

\begin{hangparas}{1.27cm}{1}
	McDonald, G. W., dan Osgood, N. D. (April 6, 2023): Agent-Based Modeling and its Tradeoffs: An Introduction \& Examples, arXiv. https://doi.org/10.48550/arXiv.2304.08497.
\end{hangparas}

\begin{hangparas}{1.27cm}{1}
	Railsback, S. F., dan Grimm, V. (2019): \textit{Agent-based an individual-based modeling: a practical introduction} (Secon edition), Princeton University Press, Princeton ; Oxford, 340.
\end{hangparas}

\begin{hangparas}{1.27cm}{1}
	Siam, M. R. K., Wang, H., Lindell, M. K., Chen, C., Vlahogianni, E. I., and Axhausen, K. (2022): An interdisciplinary agent-based multimodal wildfire evacuation model: Critical decisions and life safety, Transportation Research Part D: \textit{Transport and Environment}, \textbf{103}, 103147. https://doi.org/10.1016/j.trd.2021.103147.
\end{hangparas}

\begin{hangparas}{1.27cm}{1}
	Suheri, T., dan Viridi, S. (2019): Gravity-Driven Agent-Based Model for Simulation of Economic Growth a Point Along a Highway, \textit{IOP Conference Series: Materials Science and Engineering}, \textbf{662}(6), 062015. https://doi.org/10.1088/1757-899X/662/6/062015.
\end{hangparas}

\begin{hangparas}{1.27cm}{1}
	Wang, J., Sun, S., Fang, S., Fu, T., dan Stipancic, J. (2017): Predicting drowsy driving in real-time situations: Using an advanced driving simulator, accelerated failure time model, and virtual location-based services, \textit{Accident Analysis \& Prevention}, \textbf{99}, 321–329. https://doi.org/10.1016/j.aap.2016.12.014.
\end{hangparas}

\begin{hangparas}{1.27cm}{1}
	Wilensky, U., dan Rand, W. (2015): \textit{An introduction to agent-based modeling: modeling natural, social, and engineered complex systems with NetLogo}, The MIT press, Cambridge (Mass.).
\end{hangparas}


\setlength{\parskip}{1\baselineskip}

\end{document}
